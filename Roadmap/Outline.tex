\documentclass[10pt,a4paper]{article}
\usepackage[utf8]{inputenc}
\usepackage[T1]{fontenc}
\usepackage{amsmath}
\usepackage{amssymb}
\usepackage{graphicx}

\usepackage{outlines}
\usepackage{enumitem}
\usepackage{xspace}

\setenumerate[1]{label=\Roman*.}
\setenumerate[2]{label=\Alph*.}
\setenumerate[3]{label=\roman*.}
\setenumerate[4]{label=\alph*.}



\begin{document}

\def\QE{\textit{Quick Explainer}\xspace}
\def\NB{\textit{Notebook}\xspace}
\def\EA{\textit{Exploratory Activity}\xspace}
	
\title{Quantitative Approaches to Larval Biology \\ {\Large Selected topics and executable examples}}
	
\author{Daniel Gr\"unbaum}
\maketitle
	
\begin{outline}[enumerate]
\1 Overview 
	\2 How to use this book
	\2 Scale Models \& Non-Dimensional Numbers
		\3 \QE: Example: The Coefficient of Drag
\1 Biomechanics
	\2 Molecular Transport Mechanisms
		\3 \QE: Diffusion, conduction and viscosity
		\3 \QE: Why the Reynolds Number indicates flow
		\3 \QE: Drag forces on spheres in fluids
		\3 \NB: Calculators for spheres moving in flow
		\3 \EA: Vertical distributions of eggs 
		\3 \QE: Approximating turbulent transport with ``Gaussian Plumes''
		\3 \EA: Deposition patterns for particles in turbulent flows
		\3 \QE: Using the Sherwood Number to calculate diffusion in flow
		\3 \NB: Diffusion of mass into and out of spheres in moving water
	\2 Mechanics of Larval Swimming
		\3 \QE: Approximating early stage larval morphologies with semi-ellipsoid chimeras
		\3 \NB: Swimming of early stage larvae in still water and shear 
\1 Demography

\1 Spatial Dynamics 
	\2 Larval Transport in Estuaries
		\3 \QE: Typical Flow Patterns in Estuaries
	\2 Predicting Larval Transport by Currents and Tides
		\3 Geophysical Models of Regional-Scale Flows
		\3 \NB: LiveOcean
		\3 \NB: Connie (if still functional)
		\3 \EA: Understanding larval dispersal using geophysical models

\1 Experimental Methods
	\2 Image/video analysis
	\2 Particle Tracking: Path Reconstruction, Analysis and Velocimetry (PTV)
		\3 \QE: Video capture and analysis in Python
		\3 \NB: Acquiring videos via OpenCV in Python
		\3 \QE: Finding objects in images (``segmentation'')
		\3 \NB: Obtaining images of larvae from videos with PlanktonImageAnalysis 
		\3 \QE: Assembling trajectories of larvae and other ``particles'' 
		\3 \NB: Quantifying movements with fostrack3
		\3 \NB: Quantifying movements with Tracker3D
		\3 \QE: What is PTV?
	\2 Particle Image Velocimetry (PIV)
		\3 \QE: What is PIV?
		\3 \NB: Quantifying Flow with OpenPIV
	\2 Artificial Intelligence (AI) classification of plankton images
		\3 Quick explainer: AI classification methods: model construction from image libraries
		\3 Quick explainer: AI classification methods: model application to classify images
		\3 \NB: Creating an image library with PlanktonImageAnalysis
		\3 \NB: Constructing a classification model
		\3 \NB: Classifying images of larvae and other particles

		

%			\4 Level 4




\end{outline}































	
	
	
	
	
	
	
	
	
	
	
	
	
	
	
	
	
	
	
	
	
	
	
	
	
\end{document}