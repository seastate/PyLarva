\documentclass[10pt,a4paper]{article}
\usepackage[utf8]{inputenc}
\usepackage[T1]{fontenc}
\usepackage{amsmath}
\usepackage{amssymb}
\usepackage{graphicx}

\usepackage{outlines}
\usepackage{enumitem}
\usepackage{xspace}

\setenumerate[1]{label=\Roman*.}
\setenumerate[2]{label=\Alph*.}
\setenumerate[3]{label=\roman*.}
\setenumerate[4]{label=\alph*.}



\begin{document}

\def\QE{\textit{Quick Explainer}\xspace}
\def\NB{\textit{Notebook}\xspace}
\def\EA{\textit{Exploratory Activity}\xspace}
\def\MD{\textit{Modeling Description}\xspace}
	
\title{Quantitative Approaches to Larval Biology \\ {\Large Selected topics and executable examples}}
	
\author{Daniel Gr\"unbaum}
\maketitle
	
\begin{outline}[enumerate]
\1 Overview 
	\2 How to use this book
	\2 Scale Models \& Non-Dimensional Numbers
		\3 \QE: Example: The Coefficient of Drag
		
\1 Biomechanics
	\2 Molecular Transport Mechanisms
		\3 \QE: Diffusion, conduction and viscosity
		\3 \QE: Time and Space Scales of Diffusion
		\3 \QE: Why the Reynolds Number indicates flow
		\3 \QE: Drag forces on spheres in fluids
		\3 \NB: Calculators for spheres moving in flow
		\3 \EA: Vertical distributions of eggs 
		\3 \QE: Approximating turbulent transport with ``Gaussian Plumes''
		\3 \EA: Deposition patterns for particles in turbulent flows
		\3 \QE: Using the Sherwood Number to calculate diffusion in flow
		\3 \NB: Diffusion of mass into and out of spheres in moving water
	\2 Mechanics of Larval Swimming
		\3 \QE: Approximating early stage larval morphologies with semi-spheroid chimeras
		\3 \NB: Swimming of early stage larvae in still water and shear 
		\3 \EA: Speed, Orientation and Effects of Environmental Flows
		\3 \QE: Basics of Computer-Aided Design (CAD)
		\3 \EA: Modeling Larval Morphology Using CAD
		\3 \EA  Deriving CAD Models of Larval Morphology from Confocal Microscopy
		\3 \NB: Simulated Swimming of Complex Larval Morphologies
		
\1 Demography
	\2 Stage- and Age-Structured Populations
		\3 \QE: What Does ``Population Structure'' Mean?
		\3 \QE: Essentials of Linear Algebra I: Multiplying Vectors and Matrices
		\3 \QE: Leslie Matrix models of structured populations
		\3 \QE: A Case Study: The Atlantic Croaker
		\3 \NB: A Stage-Within-Age Model of Atlantic Croaker Life History
		\3 \EA: Population Trajectories of Atlantic Croakers Under Environmental Stress
		\3 \QE: Stable Age Distributions in Leslie Matrix Population Models
		\3 \EA: Transient \textit{vs}. Stable Age Distributions in the Croaker Model 
		\3 \QE: Metrics for Responses to Environmental Stress across Life History Stages
		\3 \NB: Calculating Elasticity Across Croaker Life History Stages
		\3 \EA: Assessing Vulnerable Life History Stages in the Atlantic Croaker   
		\3 \QE: Evolutionary Theories of Egg Size
		\3 \NB: A Conjectural Model of Egg Size Evolution in the Atlantic Croaker
		\3 \EA: Predicting Impacts of Fishing on Croaker Egg Size
	\2 Specialist and Generalist Settling Strategies in Variable Habitats
		\3 \QE: Diversity of Settling Strategies in \textit{Spirorbis borealis}
		\3 \QE: Ordinary Differential Equation (ODE) Population Models
		\3 \QE: Coexistence \textit{vs}. Competitive Exclusion in ODE Models 
		\3 \NB: A Model of Competition between \textit{Spirorbis} variants
		\3 \EA: Assessing Conditions for Coexistence and Exclusion among \textit{Spirorbis} variants
	\2 Demography of Larval Cloning
		\3 \QE: Size-Based Models of Larval Development
		\3 \QE: Larval Cloning in the Laboratory and in the Field
		\3 \QE: Partial Differential Equation Population Models
		\3 \QE: Expressions of Larval Cloning Strategies in Equation Form
		\3 \NB: A Size-Based Model of Larval Cloning
		\3 \EA: Demographic Consequences of Larval Cloning

\1 Spatial Dynamics 
	\2 Effects of Spatial Scale on Persistence and Extinction
		\3 \QE: Application of Diffusion Scaling to Random Walks
		\3 \QE: Individual-Based (or ``Agent-Based'') Models
		\3 \NB: A MESA Implementation of NetLogo's Wolf-Sheep-Grass (WSG) Model (AgentBased3trophic)
		\3 \EA: Estimating Demographic Timescales for the WSG Model
		\3 \EA: Assessing Habitat Size Effects on Persistence-Extinction Transitions
	\2 The Availability of ``Patchy'' (Spatially \& Temporally Heterogeneous) Resources 
		\3 \QE  Scaling Analysis of Foraging, Reproduction \& Consumption
		\3 \NB: Simulation Predator-Prey Interactions across Frost, Strathmann \& Lessard Numbers 
	\2 Larval Transport in Estuaries
		\3 \QE: Typical Flow Patterns in Estuaries
	\2 Predicting Larval Transport by Currents and Tides
		\3 Geophysical Models of Regional-Scale Flows
		\3 \NB: LiveOcean
		\3 \NB: Connie (if still functional -- \verb|https://connie.csiro.au/|)
		\3 \EA: Understanding larval dispersal using geophysical models

\1 Experimental Methods
	\2 Image/video analysis
	\2 Particle Tracking: Path Reconstruction, Analysis and Velocimetry (PTV)
		\3 \QE: Video capture and analysis in Python
		\3 \NB: Acquiring videos via OpenCV in Python
		\3 \QE: Finding objects in images (``segmentation'')
		\3 \NB: Obtaining images of larvae from videos with PlanktonImageAnalysis 
		\3 \QE: Assembling trajectories of larvae and other ``particles'' 
		\3 \NB: Quantifying movements with fostrack3
		\3 \NB: Quantifying movements with Tracker3D
		\3 \QE: What is PTV?
	\2 Particle Image Velocimetry (PIV)
		\3 \QE: What is PIV?
		\3 \NB: Quantifying Flow with OpenPIV
	\2 Artificial Intelligence (AI) classification of plankton images
		\3 Quick explainer: AI classification methods: model construction from image libraries
		\3 Quick explainer: AI classification methods: model application to classify images
		\3 \NB: Creating an image library with PlanktonImageAnalysis
		\3 \NB: Constructing a classification model
		\3 \NB: Classifying images of larvae and other particles

		

%			\4 Level 4




\end{outline}































	
	
	
	
	
	
	
	
	
	
	
	
	
	
	
	
	
	
	
	
	
	
	
	
	
\end{document}